\documentclass[10pt,a4paper]{article}

%
% (C) Leo Lahti 2008-2010
% 
% Licence: GNU General Public Licence 2, or at your option, a later
% version
%

%
% Tested on R281 of TKK computing center
%


\usepackage{amsmath,amssymb}
\usepackage{epsf,epsfig}
\usepackage{graphicx}

%\usepackage[T1]{fontenc}
%\usepackage[latin1]{inputenc}
\usepackage{t1enc} %utf8

\setlength{\textwidth}{450pt}
\setlength{\oddsidemargin}{0pt}
\setlength{\marginparwidth}{0pt}
\setlength{\topmargin}{0pt}
\addtolength{\textheight}{120pt}
\setlength{\leftmargin}{0cm}
\setlength{\rightmargin}{0cm}
%\linespread{1.24} %rivivali 1.5
\setlength{\parindent}{0mm}
\setlength{\parskip}{2mm}
\setlength{\voffset}{-1in} 
%\addtolength{\textheight}{40pt}

\title{Introduction to R and BioConductor}
\author{Leo Lahti}
\date{\today{}}
\usepackage{Sweave}
\begin{document}
\maketitle

%\begin{abstract}
%Short and simplistic overview of R/BioConductor with basic hands-on
%exercises. For a full description of R, see www.r-project.org.
%\end{abstract}

\section{Starting with R/BioC}

\subsubsection*{Why use R?}

See http://monkeysuncle.stanford.edu/?p=367

\subsubsection*{Introductory material}

R reference card: a useful and brief 4-page intro: 
\texttt{http://www.rpad.org/Rpad/R-refcard.pdf}

Recommended book with example source codes online:
\texttt{http://www.bioconductor.org/pub/docs/mogr/}

The project pages www.r-project.org and www.bioconductor.org provide
extensive resources.

\subsubsection*{R-Matlab dictionary}

If you already know Matlab this may help:
http://www.math.umaine.edu/\~hiebeler/comp/matlabR.pdf

\subsubsection*{Search engine for R}

Seeking R-related information from www is challenging. This solves the
problem: http://www.rseek.org/


\subsubsection*{IRC}

For online support, go to \#r-project@IRCnet (Finnish community) or
\#R@FreeNode (International channel).


\subsubsection*{Graphics}

R has excellent graphics capabilities with publication quality. See
graphics galleries for {\it examples with source code} (!):\\
http://addictedtor.free.fr/graphiques/\\
http://bm2.genes.nig.ac.jp/RGM2/


\subsubsection*{R and \LaTeX}

Automated \LaTeX document generation with parameters and plots:\\
http://www.stat.uni-muenchen.de/\~leisch/Sweave/

This document was produced with Sweave. The source (.Rnw) is provided
at course home page, and can be converted into \LaTeX code in R with
the command 'Sweave('handout.Rnw')'.

'brew' is an alternative package which allows the use of for loops in
document generation.

\subsubsection*{Graphical User Interfaces (GUIs) for R?}

Miscellaneous GUIs for R, in case you need one:

\begin{itemize}
\item Tinn-R
\item R Commander (basic-statistics GUI)
\item Rattle (data mining)
\item RKWard (KDE libraries)
\item JGR (universal Java GUI)
\item PMG
\end{itemize}

\subsection{Installing packages in Aalto/TKK computer system}

It is recommended that you install R (r-project.org) and BioConductor
(bioconductor.org) on your personal computer. 

It is often necessary to install new packages with specific
functionality. In Aalto/TKK computer system, the quickest option is to
install to your home directory (if you have enough disk space). For
example the following commands can be used to install packages within
R:

\begin{Schunk}
\begin{Sinput}
> install.packages("vabayelMix")
\end{Sinput}
\end{Schunk}

or from source tarball
\begin{Schunk}
\begin{Sinput}
> install.packages("mypackage.tar.gz", repos = NULL)
\end{Sinput}
\end{Schunk}

This works for BioConductor packages (installs 'qvalue'):
\begin{Schunk}
\begin{Sinput}
> source("http://www.bioconductor.org/getBioC.R")
> getBioC("qvalue")
\end{Sinput}
\end{Schunk}


If the package would be useful for other people (or if you want to
save your personal disk space), check if it the package has debian
source available using command line 'apt-cache search \^r-'. If a
debian package is available, request installation from
servicedesk@tkk.fi. Otherwise, send mail to leo.lahti@tkk.fi.


\section{Hands-on exercises}


Download R reference card:
\texttt{http://www.rpad.org/Rpad/R-refcard.pdf}. Start R by typing
'R'. Let's start with

\begin{Schunk}
\begin{Sinput}
> print("Hello, world!")
\end{Sinput}
\begin{Soutput}
[1] "Hello, world!"
\end{Soutput}
\end{Schunk}

Random samples from normal distribution and a histogram:
\begin{Schunk}
\begin{Sinput}
> set.seed(123)
> random.data <- rnorm(1000, mean = 0, sd = 1)
> hist(random.data)
\end{Sinput}
\end{Schunk}

(Note: {\it rnorm} generates random samples from normal
distribution. If you don't set random seed ({\it set.seed}), the code
will draw different random data at each run. By setting random seed,
you can retrieve the same random data later on, if necessary. This is
standard procedure which allows exact reproducibility of the results.)


Define title and axes:
\begin{Schunk}
\begin{Sinput}
> hist(random.data, main = "Random histogram", xlab = "X values", 
+     ylab = "Count")
\end{Sinput}
\end{Schunk}

See histogram function in more detail: type '?hist' or
'help(hist)'. 

\subsubsection*{Example data set}

Investigate example data:

\begin{Schunk}
\begin{Sinput}
> iris
> dim(iris)
> dimnames(iris)
> summary(iris)
\end{Sinput}
\end{Schunk}

Boxplot of the first four columns.

\begin{Schunk}
\begin{Sinput}
> boxplot(iris[, 1:4])
\end{Sinput}
\end{Schunk}
\includegraphics{handout-008}

Compare sepal length between two flower species with t-test:

\begin{Schunk}
\begin{Sinput}
> setosaRows = (iris[, "Species"] == "setosa")
> virginicaRows = (iris[, "Species"] == "virginica")
> x = iris[setosaRows, "Sepal.Length"]
> y = iris[virginicaRows, "Sepal.Length"]
> t.test(x, y)
\end{Sinput}
\begin{Soutput}
	Welch Two Sample t-test

data:  x and y 
t = -15.3862, df = 76.516, p-value < 2.2e-16
alternative hypothesis: true difference in means is not equal to 0 
95 percent confidence interval:
 -1.786760 -1.377240 
sample estimates:
mean of x mean of y 
    5.006     6.588 
\end{Soutput}
\end{Schunk}

\subsubsection*{Principal component analysis (PCA)}

Investigate variable contents with 'names', call subvariables with
'\$' or '@':


\begin{Schunk}
\begin{Sinput}
> iris.pca <- prcomp(iris[, 1:4])
> names(iris.pca)
\end{Sinput}
\begin{Soutput}
[1] "sdev"     "rotation" "center"   "scale"    "x"       
\end{Soutput}
\end{Schunk}

Plot variance along each principal component
\begin{Schunk}
\begin{Sinput}
> barplot(iris.pca$sdev)
\end{Sinput}
\end{Schunk}

Plot first principal component
\begin{Schunk}
\begin{Sinput}
> barplot(iris.pca$rotation[, "PC1"], main = "PCA")
\end{Sinput}
\end{Schunk}

Plot data for two random features, and then along
the most important principal components:

\begin{Schunk}
\begin{Sinput}
> par(mfrow = c(2, 1))
> plot(iris[, 1:2], main = "Two of the original features")
> plot(iris.pca$x[, 1:2], main = "Two PCA features")
\end{Sinput}
\end{Schunk}
\includegraphics{handout-013}



\subsubsection*{Plots with different colors}

\begin{Schunk}
\begin{Sinput}
> plot(iris[iris[, "Species"] == "setosa", 1:2], col = "red", xlim = range(iris[, 
+     1:2]), ylim = range(iris[, 1:2]))
> points(iris[iris[, "Species"] == "virginica", 1:2], col = "blue")
\end{Sinput}
\end{Schunk}


Order and plot correlations between 20 random samples using first 4
features:

\begin{Schunk}
\begin{Sinput}
> random.samples <- sample(nrow(iris), 20)
> flower.correlations <- cor(t(iris[random.samples, 1:4]))
> heatmap(flower.correlations, scale = "none")
\end{Sinput}
\end{Schunk}


\subsubsection*{Produce PDF}

This will save 'myFigure.pdf' in your working directory (check with
'getwd()')

\begin{Schunk}
\begin{Sinput}
> pdf("myFigure.pdf")
> heatmap(flower.correlations, scale = "none", col = grey(seq(0, 
+     1, length = 100)))
> dev.off()
\end{Sinput}
\end{Schunk}

Remember to close the output stream with 'dev.off()'!


\subsubsection*{K-means}

Based on the plot it seems that there are two groups of flowers. Use
K-means to detect these clusters. It turns out that the clusters are
explainable by flower species:

\begin{Schunk}
\begin{Sinput}
> km <- kmeans(iris[random.samples, 1:4], centers = 2)
> names(km)
> iris[names(which(km$cluster == 1)), "Species"]
> iris[names(which(km$cluster == 2)), "Species"]
\end{Sinput}
\end{Schunk}

\subsubsection*{Hierarchical clustering}


\begin{Schunk}
\begin{Sinput}
> d <- dist(iris[, 1:4])
> hc <- hclust(d, method = "ave")
> plot(hc, hang = -1)
\end{Sinput}
\end{Schunk}

Check source code of the function with 'hclust'.

\section{FAQ}

\subsubsection*{How to create multidimensional matrices?} 

\begin{Schunk}
\begin{Sinput}
> a <- array(0, dim = c(10, 4, 2))
\end{Sinput}
\end{Schunk}

\subsubsection*{How to run R scripts from source file?}

\begin{Schunk}
\begin{Sinput}
> source("myscript.R")
\end{Sinput}
\end{Schunk}

\subsubsection*{How to run R scripts on command line (batch mode)?} 

Something along these lines:\\
\#!/bin/sh\\
/your.R.path/R CMD BATCH --no-save myscript.R myscript.Rlog


\subsubsection*{Why my figure does not pop up?} 

Typically the previous output stream has been {\it opened} but {\it
not closed}. Close output stream(s) using dev.off(), this may need to
be applied several times.

\section{Advanced tips, tricks \& hacks} 

If you already know R, check the following packages:

\begin{itemize}
\item Matrix (improved matrix operations capabilities)
\item plyr (efficient data manipulation)
\item ggplot2 (high-quality graphics tools)
\item tikzdevice (R and \LaTeX graphics combined)
\item http://cran.r-project.org/web/views/HighPerformanceComputing.html (parallel computing etc.)
\end{itemize}


%system("latex handout.tex && dvipdf handout && gv handout.pdf &")


\end{document}
